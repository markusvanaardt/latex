\documentclass [10pt] {article}
\begin{document}
your text
\end{document}